\documentclass{article}

\usepackage[brazil]{babel}
\usepackage{fontspec}

\title{Modelagem e Simulação - INE 5425\\
Programa de Simulação em Linguagem de Propósito Geral}

\author{Evandro Chagas Ribeiro da Rosa}

\begin{document}

\maketitle

\section{Implementação}
A implementação foi feita em \texttt{C++11} utilizando \texttt{Qt} para interface gráfica.

\subsection{Eventos}
A gerencias dos eventos é feita pela classe \texttt{mod::Oraculo}, com os
seguintes atributos:

\begin{itemize}
  \item \texttt{tempo\_}: guarda o tempo atual da simulação
  \item \texttt{tempo\_total}: tempo que a simulação deve terminar.
  \item \texttt{events}, do tipo \texttt{std::multiset<Event>}, guarda os eventos
    ordenados pelo tempo em uma arvore \textit{RB}.
\end{itemize}

Os eventos são da classe \texttt{mod::Event} que contem:
\begin{itemize}
  \item \texttt{time\_}: tempo em que o evento deve ser executado, utilizado para atribuir 
    uma relação de ordem entre os eventos.
  \item \texttt{text}: uma descrição do evento, utilizado para \textit{debugging}.
  \item \texttt{call}: do tipo \texttt{std::function<void()>}, todo evento é um método 
    que recebe e retorna \texttt{void};
\end{itemize}

\subsubsection{Adicionar Evento} 
Para adicionar um evento basta chamar a função \texttt{mod::Oraculo::add\_event} passando o
método que sera chamado no evento (\texttt{F call}), tempo no qual o evento será chamado
(\texttt{double time}) e a descrição do evento (\texttt{std::string text}). O Evento será
criado com esse parâmetros e adiciono em \texttt{events}.

\subsubsection{Executar Eventos}
A função \texttt{mod::Oraculo::run} recebe como argumento o tempo limite de execução
(\texttt{double limit}), os eventos serão chamados em ordem em quanto não estourar
o tempo limite ou o tempo total de execução.

\subsection{Funções de probabilidade}
As funções de probabilidade são geradas pelo método \texttt{mod::parser} que recebe um 
\texttt{std::string} e retorna um \texttt{func::func (Aka std::function<double()>)}. 
A \texttt{std::string text} deve seguir os seguintes padrões:
\begin{itemize}
  \item \texttt{expo media}: Exponencial.
  \item \texttt{norm media dp}: Normal.
  \item \texttt{tria min moda max}: Triangular
  \item \texttt{unif min max} : Uniforme;
  \item \texttt{cons valor}: retorna sempre valor.
\end{itemize}

\subsection{Elementos da modelagem}
Todos elementos da modelagem estão conectados na classe \texttt{Estado}, a baixo segue a lista do elementos e por quais estatísticas são responsáveis.

\begin{itemize}
  \item \texttt{Entidade}: pode ser do \texttt{Tipo} \texttt{um} ou \texttt{dois}. Não 
    aparece na classe \texttt{Estado} por ser um elemento dinâmico transita pela simulação.
    Responsável pelas estatísticas  de ``Tempo Médio de uma Entidade na Fila'' e ``Tempo 
    Médio no Sistema''.
\end{itemize}

\end{document}

